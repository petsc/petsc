% $Id: onepager.tex,v 1.25 2000/04/14 15:02:11 curfman Exp curfman $ 
\documentstyle[epsf,sty/handpage,/home/gropp/hyper/style/anlhtext]{article}
\pagestyle{empty}
\begin{document}
\pagestyle{empty}
\title{PETSc 2.0 for MPI}
\thanks{Mathematics and Computer Science Division,
Argonne National Laboratory,
Argonne, IL 60439-4801.
This work was supported by the Mathematical,
        Information, and Computational Sciences Division subprogram of
        the Office of Computational and Technology Research,
        U.S. Department of Energy, under Contract W-31-109-Eng-38.}

\date{}
\maketitle

\newcommand{\vsp}{\vspace{-1.5mm}}

% Note:  The following classification statement was required for SC95.
%        We can zap this for other hand-outs.
% \section*{Classification} Parallel Algorithms and Software

\section*{Description}

\code{PETSc}, the Portable, Extensible Toolkit for Scientific Computation,
is a suite of uni- and parallel-processor codes that are intended for
the solution of large-scale problems modeled by partial differential
equations.  \code{PETSc} employs the MPI standard for all
message-passing communication.  The code is written in a
data-structure-neutral manner to enable easy reuse and flexibility.

\code{PETSc} integrates a hierarchy of components, thus
enabling the user to employ the level of abstraction that is most
natural for a particular problem.  Some of the components are
\vspace{-.4cm}
\begin{itemize}
\item \code{Mat} - a suite of data structures and code
      for the manipulation of parallel sparse matrices,
\vsp
\item \code{PC} - a collection of preconditioners,
\vsp
\item \code{KSP} - data-structure-neutral implementations of
      many popular Krylov subspace iterative methods,
\vsp
\item \code{SLES} - a higher-level interface for the solution of
      large-scale linear systems,
\vsp
\item \code{SNES} - data-structure-neutral implementations of Newton-like
      methods for nonlinear systems, and
\vsp
\item \code{TS} - timestepping code for the scalable solution of
      nonlinear ODEs arising from applying general ``methods of lines'' 
      techniques to time-dependent PDEs. Includes support for 
      pseudo-timestepping. 
\end{itemize}
\vsp

\code{ PETSc} is easy to use for beginners.  Moreover, its careful
design gives advanced users detailed control over the
solution process. \code{PETSc} includes an expanding suite of parallel
linear and nonlinear equation solvers that are easily used in
application codes written in C, C++, and Fortran.  \code{PETSc} also
provides many of the mechanisms needed within parallel application
codes, such as parallel matrix and vector assembly routines
that allow the overlap of communication and computation.  In addition,
\code{PETSc} includes growing support for the management of structured and
unstructured grid data.
\vsp

\section*{Applications}
\code{PETSc} is intended for use in large-scale application projects,
and several ongoing computational science projects at Argonne and
other institutions are built around these tools.  
One such application for a fully unstructured, fully implicit Euler
solver was a "special catagory" winner in the 1999 Gordon Bell
competition at SC99 with a rate of over 225 Gflops on the ASCI
Red machine. 
With strict attention to component interoperability, \code{PETSc}
facilitates the integration of independently developed application
modules, which often most naturally employ different coding styles and
data structures. 

% This code also achieved a 95\% efficiency in going from
% 256 to 2048 processors and over 70 Gflops on 1280 processors of the
% Cray T3E.

\section*{Availability}

\code{PETSc} package is freely available and supported.
The \code{PETSc} distribution contains all source code, 
a users guide, manual pages, and a
collection of examples. There are over 250 members of the \code{PETSc}-users mailing list.

\section*{Computational Environment}
\code{PETSc} is available for Linux, Sun, DEC Alpha, Silicon Graphics, HP-UX, Windows NT/9X, 
and IBM workstations; the Cray T3E; the IBM SP; the SGI Origin; and the HP Exemplar.

\vspace{-.1cm}
\contact{Satish Balay, William Gropp, Lois Curfman McInnes, Barry Smith}
\email{\bf{http://www.mcs.anl.gov/petsc}}
\phone{http://www.mcs.anl.gov/petsc}
\makeinfo
\end{document}
