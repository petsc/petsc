
\chapter{Adding a solver}
\label{chapter:addsolver}

\section{Adding a Solver to TAO}

New optimization solvers can be added to TAO.  TAO provides tools for
facilitate the implementation of a solver.  The advantages of implementing
a solver using TAO are several.

\begin{enumerate}
\item TAO includes other optimization solvers with an identical interface, 
so application problems may
conveniently switch solvers to compare their effectiveness.

\item TAO provides support for function evaluations and
derivative information.  It allows for the direct evaluation
of this information by the application developer, 
and contains limited support for finite difference, and
allows the uses of matrix-free methods.
The solvers can obtain this function and derivative information
through a simple interface  while the details of its computation 
are handled within the toolkit.

\item TAO provides line searches, convergence
tests, monitoring routines, and other tools
which are helpful within an optimization algorithm.
The availability
of these tools means that the developers of the optimization
solver do not have to write these utilities.

\item PETSc offers vectors, matrices, index sets, and linear solvers
that can be used by the solver.  These objects are standard mathematical
constructions that have many different implementations.
The objects may be distributed over multiple processors, restricted to
a single processor, have a dense representation, 
use a sparse data structure, or vary in many other ways.  
TAO solvers do not need to know how
these objects are represented or how the operations defined on them
have been implemented.  Instead, the solvers apply these operations
through an abstract interface that leaves the details to PETSc
and external libraries.
This abstraction allows solvers to work seamlessly with a variety
of data structures while allowing application developers to select 
data structures tailored for their purposes.

\item TAO supports an interface to PETSc and
allows the integration of other libraries as well.
When used with PETSc, TAO provides the user a convenient
method for setting options at runtime, performance profiling, and debugging.

\end{enumerate}

\section{TAO Interface with Solvers}
TAO solvers must be written in C or C++ and include several routines with
a particular calling sequence.  Two of these routines are mandatory:
one that initializes the TAO structure with the appropriate information
and one that applies the algorithm to a problem instance.
Additional routines may be written to set some options within the
solver, view the solver, setup appropriate data structures, and destroy
these data structures.
In each of these routines except the initialization routine, 
there are two arguments.  

The first argument
is always the TAO structure.  This structure may be used to obtain the
vectors used to store the variables and the function gradient, evaluate
a function and gradient, solve a set of linear equations, perform a line 
search, and apply a convergence test.

The second argument is specific to this solver.  This pointer will be set
in  the initialization routine and cast to an appropriate type in
the other routines.  To implement the conjugate
gradient algorithm, for instance, the following structure may
be useful.
\begin{verbatim}
typedef struct{

  PetscReal beta;

  Vec dx;   /* step direction */
  Vec ww;   /* work vector    */

} TAO_CG;
\end{verbatim}
This structure contains two work vectors and a scalar.  Vectors
for the solution and gradient are not needed here because the TAO
structure has pointers to them.

\subsection{Solver Routine}
All TAO solvers have a routine that accepts a TAO structure and
computes a solution.  
TAO will call this routine when the application
program uses the routine {\tt TaoSolverSolve()} and pass to the solver
information
about the objective function and constraints, pointers to the
variable vector and gradient vector, and support for line searches,
linear solvers, and convergence monitoring.  As an example, consider
the following code which solves an unconstrained minimization problem
using the conjugate gradient method.

\begin{verbatim}
PetscErrorCode TaoSolverSolve_CG(TaoSolver tao, void *solver){

  TAO_CG  *cg = (TAO_CG *) solver;
  Vec x = tao->solution;
  Vec g = tao->gradient;
  Vec s = tao->stepdirection;
  PetscInt     iter=0;
  PetscReal  gnormPrev,gdx,f,gnorm,steplength=0;
  TaoLineSearchTerminationReason lsflag=TAO_LINESEARCH_CONTINUE_ITERATING;
  TaoSolverTerminationReason reason=TAO_CONTINUE_ITERATING;
  PetscErrorCode ierr;

  PetscFunctionBegin;

  ierr = TaoSolverComputeObjectiveAndGradient(tao,x,&f,g);CHKERRQ(ierr);
  ierr = VecNorm(g,NORM_2,&gnorm);  CHKERRQ(ierr);

  ierr = VecSet(s,0); CHKERRQ(ierr); 

  cg->beta=0;
  gnormPrev = gnorm;

  /* Enter loop */
  while (1){

    /* Test for convergence */
    ierr = TaoSolverMonitor(tao,iter,f,gnorm,0.0,step,&reason);CHKERRQ(ierr);
    if (reason!=TAO_CONTINUE_ITERATING) break;

    cg->beta=(gnorm*gnorm)/(gnormPrev*gnormPrev);
    ierr = VecScale(s,cg->beta); CHKERRQ(ierr);
    ierr = VecAXPY(s,-1.0,g); CHKERRQ(ierr);
    
    ierr = VecDot(s,g,&gdx); CHKERRQ(ierr);
    if (gdx>=0){     /* If not a descent direction, use gradient */
      ierr = VecCopy(g,s); CHKERRQ(ierr);
      ierr = VecScale(s,-1.0); CHKERRQ(ierr);
      gdx=-gnorm*gnorm;
    } 

    /* Line Search */
    gnormPrev = gnorm;  step=1.0;
    ierr = TaoLineSearchSetInitialStepLength(tao->linesearch,1.0);
    ierr = TaoLineSearchApply(tao->linesearch,x,&f,g,s,&steplength,&lsflag);
    ierr = VecNorm(g,NORM_2,&gnorm);CHKERRQ(ierr);
    iter++;
  }
  
  PetscFunctionReturn(0);
}
\end{verbatim}

The first line of this routine cast the second argument to a pointer
to a {\tt TAO\_CG} data structure.  This structure contains pointers
to three vectors and a scalar which will be needed in the algorithm.

After declaring an initializing several variables, the solver lets TAO 
evaluate the function and gradient at the
current point in the using the routine 
{\tt TaoSolverComputeObjectiveAndGradient()}.
Other routines may be used to evaluate the Hessian matrix or evaluate
constraints.  TAO may obtain this information using direct evaluation 
of other means, but the these details do not affect our implementation
of the algorithm.

The norm of the gradient is a standard measure used
by unconstrained minimization solvers to define convergence.
This quantity is always nonnegative and equals zero at the solution.  
The solver will pass this quantity, the current
function value, the current iteration number, and a measure of
infeasibility to TAO with the routine
\begin{verbatim}
   PetscErrorCode TaoSolverMonitor(TaoSolver,PetscInt,PetscReal,double,double,double,
                  TaoTerminateReason*);
\end{verbatim}
Most optimization algorithms are iterative in nature, and solvers should
include this command somewhere in each iteration.  This routine
records this information, applies any monitoring routines and 
convergence tests set by default or the user.

In this routine, the second argument is the current
iteration number, and the third argument is the current function value.
The fourth argument is a nonnegative error measure associated with the
distance between the current solution and the optimal solution.  Examples
of this measure are the norm of the gradient or the square root of a duality 
gap. The fifth measure is a nonnegative error 
that is nonnegative and usually
represents a residual between the current function value and the
optimal solution, such as the norm of the
gradient.  The sixth argument is a nonnegative steplength, 
or the multiple of the step direction added to the previous iterate.
The results of the convergence test are returned in the last argument.
If the termination reason is {\tt TAO\_CONTINUE\_ITERATING}, the
algorithm should continue.

After this monitoring routine, the solver computes a step direction
using methods defined on the TaoVec object.  These methods include
adding vectors together and computing an inner product.  A full list
of these methods can be found in the manual pages.

Nonlinear conjugate gradient algorithms also require a line search.  TAO
provides several line searches and support for using them.
The routine
\begin{verbatim}
   int TaoLineSearchApply(TAO_SOLVER tao, TaoVec *xx, TaoVec *gg, TaoVec *dx,
                          TaoVec *ww, double *f, double *step,
                          int*flag)
\end{verbatim}
passes the current solution, gradient, and objective value to the
solver and returns a new solution, gradient, and objective value.  More
details on line searches can be found in the Section~\ref{sec:TaoLineSearch}
The details of this line search are should be specified elsewhere, when
the line search is created.

TAO also includes support for linear solvers.  Although this algorithm
does not require one, linear solvers are an important part of many
algorithms.  Details on the use of these solvers can be found in
Section~\ref{sec:TaoLinearSolvers}.

\subsection{Creation Routine}
The TAO solver is initialized to for a particular algorithm in a separate
routine.  This routine sets default convergence tolerances, creates a
line search or linear solver if needed, and creates structures needed
by this solver.   For example, the routine that creates the nonlinear
conjugate gradient algorithm shown above can be implemented as follows.
\begin{verbatim}
EXTERN_C_BEGIN
int TaoCreate_CG_FR(TAO_SOLVER tao)
{
  TAO_CG *cg;
  int    info;

  TaoFunctionBegin;

  info = TaoNew(TAO_CG,&cg); CHKERRQ(info);

  info = TaoSetMaximumIterates(tao,2000); CHKERRQ(info);
  info = TaoSetTolerances(tao,1e-4,1e-4,0,0); CHKERRQ(info);
  info = TaoSetMaximumFunctionEvaluations(tao,4000); CHKERRQ(info);

  info = TaoCreateMoreThuenteLineSearch(tao,0,0.1); CHKERRQ(info);

  info = TaoSetTaoSolveRoutine(tao,TaoSolve_CG_FR,(void*)cg); CHKERRQ(info);
  info = TaoSetTaoSetUpDownRoutines(tao,TaoSetUp_CG,TaoDestroy_CG); CHKERRQ(info);
  info = TaoSetTaoOptionsRoutine(tao,TaoSetOptions_CG_FR); CHKERRQ(info);
  info = TaoSetTaoViewRoutine(tao,TaoView_CG); CHKERRQ(info);

  TaoFunctionReturn(0);
}
EXTERN_C_END
\end{verbatim}
The first thing this routine does after declaring some variables, is allocate
memory for the {\tt TAO\_CG} data structure.  Clones of the the 
variable vector assed into TAO in the {\tt TaoCreate()} routine 
are used as the two work vectors.
This routine also sets some default convergence tolerances and creates
a particular line search.
These defaults could be specified in the routine that solves the problem,
but specifying them here gives the user the opportunity to modify these
parameters.

Finally, this solvers passes to TAO the names of all the other routines
used by the solver.  

Note that the lines {\tt EXTERN\_C\_BEGIN} and {\tt EXTERN\_C\_END} surround
this routine.  These macros are required to preserve the name of this
function without any name-mangling from the C++ compiler.

\subsection{Destroy Routine}
Another routine needed by most solvers destroys the data structures
creates by earlier routines.  For the nonlinear conjugate gradient
method discussed earlier, the following routine destroys the two
work vectors, the line search, and the {\tt TAO\_CG} structure. 
\begin{verbatim}
int TaoDestroy_CG(TAO_SOLVER tao, void *solver)
{
  TAO_CG *cg = (TAO_CG *) solver;
  int    info;

  TaoFunctionBegin;

  info = TaoVecDestroy(cg->gg); CHKERRQ(info);
  info = TaoVecDestroy(cg->ww);CHKERRQ(info);
  info = TaoVecDestroy(cg->dx);CHKERRQ(info);

  info = TaoLineSearchDestroy(tao);CHKERRQ(info);
  TaoFree(cg);

  TaoFunctionReturn(0);
}
\end{verbatim}
Other algorithms may destroy matrices, linear solvers, index sets, or
other objects needed by the solver.  This routine is called from within
the {\tt TaoDestroy()} routine.


\subsection{SetUp Routine}
If this routine has been set by the initialization routine, TAO
will call it during the {\tt TaoSetApplication()}.  
This routine is optional, but is often a used to allocate
the gradient vector, work vectors, and other data structures 
required by the solver.
It should have the form
\begin{verbatim}
int TaoSetUp_CG(TAO_SOLVER,void*);
{
  int    info;
  TaoVec *xx;
  TaoFunctionBegin;

  info = TaoGetSolution(tao,&xx);CHKERRQ(info);
  info = xx->Clone(&cg->gg); CHKERRQ(info);
  info = xx->Clone(&cg->ww); CHKERRQ(info);
  info = xx->Clone(&cg->dx); CHKERRQ(info);
  TaoFunctionReturn(0);
}
\end{verbatim}
The second argument can be cast to the appropriate data structure.
Many solvers use a similar routine to allocate data structures
needed by the solver but not created by the initialization routine.


%%% Local Variables: 
%%% mode: latex
%%% TeX-master: "manual_tex"
%%% End: 
