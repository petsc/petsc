% $Id: book.tex,v 1.1 1995/11/02 13:36:35 bsmith Exp bsmith $ 
%
%  Numerical Analysis and Scientific Computing
%
\documentstyle[psfig,../sty/verbatim,../sty/here]{report} 
\setlength{\textwidth}{6.5in}
\setlength{\oddsidemargin}{0.0in}
\setlength{\evensidemargin}{0.0in}
\setlength{\textheight}{9.2in}
\setlength{\topmargin}{-.8in}

\newcommand{\sindex}[1]{\index{#1}}
\newcommand{\F}{\mbox{\boldmath \(F\)}}
\newcommand{\x}{\mbox{\boldmath \(x\)}}
\newcommand{\rr}{\mbox{\boldmath \(r\)}}

\makeindex
 

% Print DRAFT in large letters across every page
\special{!userdict begin /bop-hook{gsave 200 70 translate
65 rotate /Times-Roman findfont 216 scalefont setfont
0 0 moveto 0.95 setgray (DRAFT) show grestore}def end}

\begin{document}

\title{{\LARGE Numerical Analysis and Scientific Computing:} \\
       {\Large Software, Algorithms, and Mathematics}\\
       {The PETSc Way}\\
      }

\author{
       \and
         Lois Curfman McInnes\thanks{Argonne National Laboratory,
         9700 South Cass Ave., Argonne, IL 60439-4844,
         curfman@mcs.anl.gov
         }
       \and
         Barry F. Smith\thanks{Argonne National Laboratory,
         9700 South Cass Ave., Argonne, IL 60439-4844,
         bsmith@mcs.anl.gov
         }
       }


\maketitle

\date{\today}

\newpage

\hbox{ }

\vspace{2in}

\noindent {\bf Introduction:} 
With the advent of superfast RISC processors and massively parallel
computers, the focus of numerical analysis has shifted and
expanded. More of the emphasis is on large scale calculations
involving hundreds of thousands or millions of unknowns and
complicated applications that involve dozens of different numerical
proceedures. Traditional numerical analysis, including algorithm
development, error analysis, convergence analysis, complexity and
stability are still important topics but they must be supplemented by
an understanding of RISC processor technology, parallel computing,
software engineering, scalability, and even application specific 
information. 

This book merges these topics into a coherent introduction to
all aspects of modern numerical analysis. We define numerical analysis
to be the study of all aspects of numerical computing. 

This book focuses on numerical analysis for the large scale solution of 
PDEs. Hence we omit many important areas of numerical analysis including:
approximation theory, constrained optimization, ODEs,...

The software package PETSc is used to explain and motivate the material.
Many of the numerical algorithms and software engineering techniques
introduced in this book are contained in the PETSc software. 

Readers of this book should have a good background in multivariant Calculus
and linear algebra. An elementary understanding of partial differential 
equations would be useful but is not required. We will introduce the material
on partial differential equations, as needed.

The use of abstractions is a very powerful technique to organize a lot of 
similar, but slightly, different concepts or relationships. Those who understand
the material benefit greatly from its use, a draw back is that those unfamilar 
with the material often can grasp the utility of abstraction used in the 
field. This book uses lots of abstracts, however they are all carefully introduced
through simple examples and in addition, we explain why the abstractions are 
appropriate and powerful.

\vspace{1in}

\noindent {\bf Acknowledgments:}
Bill Gropp and anyone else who is nice to us.

\medskip \medskip 

\pagenumbering{roman}
\setcounter{page}{3}
\tableofcontents
\clearpage
\pagenumbering{arabic}


\chapter{Review of Linear Algebra} 
\label{chapter:rola}

\subsection{Introduction}
This book is concerned with the computing, algorithmic and numerical 
aspects of the solution of partial differential equations and large
scale unconstrained minimization. Many of the large scale applications 
can be posed in this form. The underlying problems are generally nonlinear.
Unfortunately, nonlinear problems are invariably difficult and cannot be 
solved directly. Rather, the nonlinear problems are solved indirectly by 
solving a series of linear problems. Fortunately, the formulation and solution
of linear problems is reasonably well understood and in many circumstances
they may be solved robustly and efficiently. 

Since linear problems form the cornerstone of nonlinear problems,
every numerical analysist must have a solid grasp of numerical methods
for linear problems. Thus this book devotes its first few chapters for 
a detailed discussion of numerical techniques and software for linear problems.
The first Chapter introduces the mathematical language for linear problems, linear
algebra. 


\section{Vector Spaces: Our most basic objects}


We need a systematic way of organizing quantities, the number of toasters sold 
everyday, or the hourly tempature. For a single quantity we can represent the 
value with a single number, but in general we need to store a list of numbers. So,
for example, the hourly tempature for the past four hours could be represented
as $ ( 70, 68, 69, 67). $

Try Again: 

We begin by defining a vector as a numerical representation of some quantities
store in a list. For example, the hourly tempature for the past three hours could 
be represented as $ ( 70, 68, 69). $ The interpretations of this object are 
virtually unlimited.

Try again:

The most elementary object of interest to numerical analysists are, of course, 
real numbers. Since we assume that all readers are familar with the mathematical
properties of real numbers we forgo all disscusions on them. In \ref{chapter:itse}




\subsection{Vectors in $R^n$}

Vectors, vector spaces, norms, inner products, subspaces

\subsection{Abstract Vector Spaces}

\section{Linear Operators: Our most basic operation}

\subsection{Matrices}
linear operators, matrices, projections, linear systems, eigenvalues

\subsection{Abstract Linear Operators}



\chapter{Introduction to Computing}
\label{chapter:itse}
RISC, cache effects, counting loads and stores, SMP, distributed memory, MPI, counting
flops.

\begin{figure}
Figures of flop speed, memory speed and ratio as a function of historical time
\end{figure}

software engineering, object oriented, data encapsulation, 
polymorphism, in heritence

\chapter{Direct Methods for Linear Systems}
\label{chapter:dmftsols}

dense: cholesky, LU, QR (sequential, parallel) level 3 BLAS
sparse: cholesky, LU, blocked forms, multifrontal

\chapter{Iterative Methods for Linear Systems}
\label{chapter:dmftsols}

KSP,
preconditioners: ilu, block Jacobi, MG, DD

\chapter{Solution of Nonlinear Systems}
\label{chapter:sons}

\chapter{Eigenvalue Problems}
\label{chapter:ep}

\chapter{Review of PDES?}


\chapter{Finite Difference Methods}
\label{chapter:fdm}

\chapter{Finite Element Methods}
\label{chapter:fem}

\section{Griding}


\vfill
\eject

\addcontentsline{toc}{chapter}{Subject Index}
\include{sindex}

\bibliographystyle{plain}
\addcontentsline{toc}{chapter}{Bibliography}
\bibliography{petsc}

\end{document}

