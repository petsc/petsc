%
%
\documentclass[twoside,11pt]{../sty/report_petsc}

\usepackage{makeidx,xspace}
\usepackage[bookmarksopen,colorlinks]{hyperref}
\usepackage[all]{hypcap}
\usepackage{color}
\input pdfcolor.tex     

\usepackage[pdftex]{graphicx}


\usepackage{times}
\usepackage{listings}
%\usepackage{psfig}
\usepackage{../sty/verbatim}
\usepackage{../sty/tpage}
\usepackage{../sty/here}
\usepackage{../sty/anlhelper}
\usepackage[hyphens,spaces,obeyspaces]{../sty/trl}
 
\setlength{\textwidth}{6.5in}
\setlength{\oddsidemargin}{0.0in}
\setlength{\evensidemargin}{0.0in}
\setlength{\textheight}{9.2in}
\setlength{\topmargin}{-.8in}

\newcommand{\findex}[1]{\index{#1}}
\newcommand{\sindex}[1]{\index{#1}}
\newcommand{\A}{\mbox{\boldmath \(A\)}}
\newcommand{\F}{\mbox{\boldmath \(F\)}}
\newcommand{\J}{\mbox{\boldmath \(J\)}}
\newcommand{\x}{\mbox{\boldmath \(x\)}}
\newcommand{\bb}{\mbox{\boldmath \(b\)}}
\newcommand{\rr}{\mbox{\boldmath \(r\)}}
hyperbaseurl

\makeindex
 
% Defines the environment where design issues are discussed. In the manual
% version of this report, these regions are ignored.
\def\design{\medskip \noindent Design Issue:\begin{em}}
\def\enddesign{\end{em} \medskip}
% Manual version:
% \def\design{\comment}
% \def\enddesign{\endcomment}

% Print DRAFT in large letters across every page
%\special{!userdict begin /bop-hook{gsave 200 70 translate
%65 rotate /Times-Roman findfont 216 scalefont setfont
%0 0 moveto 0.95 setgray (DRAFT) show grestore}def end}

% Defines that we're doing the whole manual, not the short intro part,
% used in part1.tex.
\def\shortintro{false}

\usepackage{fancyhdr,lastpage}
\pagestyle{fancy}
\rhead{PETSc 3.2 \today}

\begin{document}

%\pagestyle{empty}
%\begin{figure*}[hbt]
%\centerline{\includegraphics{titlepage1}}
%\end{figure*}

%  ANL changed the style of its title pages - so we are using titlepage1.pdf [above]
%  The following is an attempt to emulate titlepage1.pdf in latex

%%%%%%%%%%%%%%%%%%%%%%%%%%%%%%%%%%%%%%%%%%%%%%%%%%%%%%%%%%%%%%%%%%%%%%%%%%%%%%%%%%%%

\hfill {\large{\bf ANL-95/11}}

\vspace*{3in}
\noindent {\huge{\bf PETSc Users Manual}}
\vspace*{8pt}
\hrule
\vspace*{8pt}
\noindent {\huge{\it Revision 3.2}}

\vspace*{1in}
\noindent by \\
S. Balay, J. Brown, K. Buschelman, V. Eijkhout, W. Gropp, D. Kaushik, \\
M. Knepley, L. Curfman McInnes, B. Smith, and H. Zhang \\
Mathematics and Computer Science Division, Argonne National Laboratory

\vspace*{10pt}
\noindent Sept 2011

\vspace*{20pt}
\noindent This work was supported by the Office of Advanced Scientific Computing Research, \\
Office of Science, U.S. Department of Energy, under Contract DE-AC02-06CH11357.

%%%%%%%%%%%%%%%%%%%%%%%%%%%%%%%%%%%%%%%%%%%%%%%%%%%%%%%%%%%%%%%%%%%%%%%%%%%%%%%%%%%%

\begin{figure*}[hbt]
\centerline{\includegraphics{titlepage2}}
\caption{}
\end{figure*}


\cleardoublepage
%\pagestyle{plain}

\vspace{1in}
\date{\today}

% Abstract for users manual
\addcontentsline{toc}{chapter}{Abstract}
% Abstract for PETSc Users Manual

%
%   Next line temp removed
%
\noindent {\bf Abstract:} 

\medskip \medskip
This manual describes the use of PETSc for the numerical solution
of partial differential equations and related problems 
on high-performance computers.  The
Portable, Extensible Toolkit for Scientific Computation (PETSc) is a
suite of data structures and routines that provide the building
blocks for the implementation of large-scale application codes on parallel
(and serial) computers.  PETSc uses the MPI standard for all
message-passing communication.

PETSc includes an expanding suite of parallel linear, nonlinear
equation solvers and time integrators that may be
used in application codes written in Fortran, C, C++, Python, and MATLAB (sequential).  PETSc
provides many of the mechanisms needed within parallel application
codes, such as parallel matrix and vector assembly routines. The library is
organized hierarchically, enabling users to employ the level of
abstraction that is most appropriate for a particular problem. By
using techniques of object-oriented programming, PETSc provides
enormous flexibility for users.

PETSc is a sophisticated set of software tools; as such, for some
users it initially has a much steeper learning curve than a simple
subroutine library. In particular, for individuals without some
computer science background, experience programming in C, C++ or Fortran and experience using a debugger such as \trl{gdb} or \trl{dbx}, it
may require a significant amount of time to take full advantage of the
features that enable efficient software use.  However, the power of
the PETSc design and the algorithms it incorporates may make the efficient
implementation of many application codes simpler than ``rolling
them'' yourself.
\begin{itemize}
\item  For many tasks a package such as MATLAB is often the best tool; PETSc is not
intended for the classes of problems for which effective MATLAB code
can be written. PETSc also has a MATLAB interface, so portions of your code can be written in MATLAB to ``try out'' the PETSc solvers. 
The resulting code will not be scalable however because currently MATLAB is inherently not scalable.
\item PETSc should not be used to attempt to provide
a ``{\bf parallel linear solver}'' in an otherwise sequential code.
Certainly all parts of a previously sequential code need not be parallelized but the 
matrix generation portion must be parallelized to expect any kind of reasonable performance.
Do not expect to generate your matrix sequentially and then ``use PETSc'' to solve
the linear system in parallel.
\end{itemize}

Since PETSc is under continued development, small changes in usage and
calling sequences of routines will occur.  PETSc is supported; see the
web site \href{http://www.mcs.anl.gov/petsc}{http://www.mcs.anl.gov/petsc} for information on
contacting support.

A \href{http://www.mcs.anl.gov/petsc/publications}{http://www.mcs.anl.gov/petsc/publications} may be found 
a list of publications and web sites that feature work involving PETSc.


We welcome any reports of corrections for this document.

\medskip \medskip





\cleardoublepage


% ---------------------------------------------------------------------------
%

\medskip\medskip

\noindent {\bf Getting Information on PETSc:} 

\medskip

\leftline{%
 \pdfstartlink user{%
        /Subtype /Link
        /A << 
            /Type /Action 
            /S /URI 
            /URI (http://www.mcs.anl.gov/petsc/docs/) 
        >>}%
    \Red  \trl{http://www.mcs.anl.gov/petsc/docs/} \Black
    \pdfendlink}

\noindent {\bf On-line:}
\begin{list}{$\bullet$}
{
\setlength{\itemsep}{-.020in} 
\setlength{\topsep}{0in} 
\setlength{\partopsep}{0in}
}
\item Manual pages for all routines, including example usage
\begin{list}{$\bullet$}
{
\setlength{\itemsep}{-.020in} 
\setlength{\topsep}{0in} 
\setlength{\partopsep}{0in}
}
   \item \trl{docs/index.html} in the distribution or 
   \item \trl{http://www.mcs.anl.gov/petsc/docs/}
\end{list}
\item Troubleshooting
\begin{list}{$\bullet$}
{
\setlength{\itemsep}{-.020in} 
\setlength{\topsep}{0in} 
\setlength{\partopsep}{0in}
}
   \item \trl{docs/troubleshooting.html} in the distribution or
   \item \trl{http://www.mcs.anl.gov/petsc/docs/troubleshooting.html}
\end{list}
\end{list}

\medskip
\noindent {\bf In this manual:}
\begin{list}{$\bullet$}
{
\setlength{\itemsep}{-.02in} 
\setlength{\topsep}{.02in} 
\setlength{\partopsep}{0in}
}
\item Basic introduction, page \pageref{sec:gettingstarted}
\item Assembling vectors, page \pageref{sec:vecbasic}; and matrices, \pageref{chapter:matrices}
\item Linear solvers, page \pageref{ch:sles}
\item Nonlinear solvers: \pageref{chapter:snes}
\item Timestepping (ODE) solvers: \pageref{chapter:ts}
\item Index, page \pageref{sec:index}
\end{list}

% ---------------------------------------------------------------------------


\medskip \medskip

\cleardoublepage

% Acknowledgements for users manual
% Acknowledgements for PETSc 2.0 Users Manual

\noindent {\bf Acknowledgments:}

\medskip \medskip 
We thank Victor Eijkhout for his valuable comments on this
manual as well as on the source code for PETSc 2.0.  We also thank David
Keyes for his insightful suggestions about increased functionality.
In addition, we thank all PETSc users for
their suggestions, bug reports, support, and encouragement.

\vspace{.3in}
Some of the source code and utilities in PETSc (or software used by PETSc)
has been written by 
\begin{itemize}
  \item Cameron Cooper, Fall 1995, (Portions of the VecScatter routines), 
  \item Matt Hille, Summer 1995, (PetscView and PetscOpts), 
  \item Peter Mell, Summer 1995, (Portions of the DA-distributed array routines),
  \item Wing-Lok Wan, Summer 1995, (the ILU portion of BlockSolve95)
\end{itemize}
while visiting Argonne National Laboratory or working with us.

\vspace{.3in}
PETSc uses routines from 
\begin{itemize}
  \item BLAS, 
  \item LAPACK,
  \item LINPACK,      (matrix factorization and solve; converted to C using f2c and then 
                      hand-optimized for small matrix sizes),
  \item MINPACK,      (matrix coloring routines for finite difference Jacobian evaluations;
                      converted to C using f2c),
  \item SPARSPAK,     (matrix reordering routines, converted to C using f2c),
to provide a small subset of its low-level functionality.
\end{itemize}

\vspace{.3in}
PETSc interfaces to the following external software
\begin{itemize}
  \item BlockSolve95, (for parallel ICC(0) and ILU(0) preconditioning),
  \item SPAI,         (for parallel sparse approximate inverse preconditiong),
  \item ESSL,         (IBM's math library for fast sparse direct LU factorization),
  \item Matlab,       (through a socket interface for graphics and numerical post processing 
                       of data),
  \item VRML,         (for simple three dimensional visualization post-processing).
\end{itemize}
These are optional packages and do not need to be installed to use PETSc.




% Blank page makes double sided printout look bettter.

\cleardoublepage

\tableofcontents

% --------------------------------------------------------------------
%                            PART 1
% --------------------------------------------------------------------
\cleardoublepage
\part{Introduction to PETSc}
\label{part_intro}
\cleardoublepage
\chapter{Getting Started}
\input{part1tmp.tex}

% --------------------------------------------------------------------
%                            PART 2
% --------------------------------------------------------------------
\cleardoublepage
\part{Programming with PETSc}
\label{part_usage}
\input{part2tmp.tex}


%------------------------------------------------------------------


\cleardoublepage
\bibliographystyle{plain}
\addtocounter{chapter}{1}
\addcontentsline{toc}{chapter}{Bibliography}
\label{sec:bib}
\bibliography{../petsc,../petscapp}

\pagestyle{empty}
\begin{figure*}[hbt]
\centerline{\includegraphics{endpage}}
\caption{}
\end{figure*}

\end{document}


